% IMPORTANT: add or remove (comment out) the boolean '\solutiontrue' below to
% create the solution document or the exercise document respectively.
% First we create the switch to make either the exercises or the solutions
\newif\ifsolution\solutionfalse
% To create the solution uncomment '\solutiontrue'
\solutiontrue

\documentclass[a4paper,11pt]{article}


\usepackage[T1]{fontenc}
\usepackage{ae, aecompl}
\usepackage{a4wide}
\usepackage{boxedminipage}
\usepackage{url}
\usepackage{graphicx}
\usepackage{enumerate}
\usepackage{hyperref}
\usepackage{listings}
\usepackage{ucs}
\usepackage[utf8x]{inputenc}
\usepackage[english]{babel}

\title{System Security,
\ifsolution Solution \else \fi
RSA and Sensor Node}
\author{Andrei Pârvu}

../author.tex

% Some useful commands and environments
\usepackage{framed}
\newenvironment{solution}%
{\par{\noindent\small\textit{Solution:}}\vspace{-12pt}\begin{framed}}%
{\end{framed}\par}

\begin{document}
\maketitle

\section{Why does RSA work?}
Given plain-text~\emph{P}, an RSA Key pair~(public key (\emph{E},\emph{N}) and
private key (\emph{D}, \emph{N})) and cipher-text~\emph{C}, what are the
mathematical expressions that describe RSA encryption and decryption, i.e.,
what is the relationship between \emph{P} and \emph{C}? Prove that the
equation for decryption is correct, i.e., detail the mathematical theorems
that result in this equation.
\ifsolution\begin{solution}
$N$ has been chosen so that it is a product of two large primes, ($N = P * Q$).
Then, $E$ is selected as being a random number, with $E$ less than and co-prime with
$\phi(N) = (P - 1) * (Q - 1)$. $(E, N)$ forms the public key.\\
The private key will be the pair $(D, N)$ , with $D$ being chosen such that
$E * D \equiv 1 \pmod{\phi(N)}$.\\
Given these properties, we can analyze the encryption and decryption.
\begin{itemize}
  \item encryption: given plaintext $p$, ciphertext $c$ will be $c = (p^E) \% N$.
  \item decryption: we compute $(c^D) \% N$, which will be equal to $(p^{E * D} \% N) =
  (p^{k * \phi(N) + 1}) \% N = ((p^{k * \phi(N)}) \% N) * (p \% N)) \% N$. Now, given
  that $p$ and $N$ are co-prime, we can use Euler Theorem \cite{eulerth} which states
  that $a^{\phi(n)} \equiv 1 \pmod{n}$. Hence, $(c^D) \% N = p \% N = p$, so we recovered
  the plaintext from the ciphertext.
\end{itemize}

\end{solution}\fi

\section{A JTAG-based attack}
In this week's exercise, the goal is to find the private key by analyzing the
firmware image. This image was obtained from a sensor node through its JTAG
interface. The goal of this exercise is to also give you some insight into JTAG
interfaces, their use and security implications.

The necessary information can be found in the \texttt{sensor} directory.

We assume that we already know the public key of the node: the public exponent
$e$ is {\tt 0x5abb} and the modulus $n$ is {\tt 0x1d7 777c 3886 3aec 21ba 2d91
ee0f af51}. But are we able to find the private key from the firmware image in
rsa.txt?

\begin{enumerate}[(a)]

\item Please read about JTAG. How does it work? What is its intended use? What
are the security implications of an active JTAG interface when an attacker has
direct access to a device?

\item What is the format of the dumped image? Is it easy to guess the key
directly from this format?

\emph{Hint 1: The format is typical for programming microcontrollers, EPROMs,
etc.}
\\
\emph{Hint 2: You might need to extract the assembly code instructions by converting
it to a msp430 binary and then disassemble it.}
\\
\emph{Hint 3: Remember that the image was generated by a memory dump. It
contains a data and a text section. However, the disassembler might not be able
to tell the difference.}

\ifsolution\begin{solution}
The dumped image file is in \emph{ihex} format \cite{ihex}. Each line in the file
has the following structure:
\begin{itemize}
  \item start colon "$:$"
  \item byte count (in our case it's {\tt 0x10})
  \item address, consisting of four hex digits
  \item record type (in our case {\tt 0x00})
  \item 16 bytes
  \item 1 checksum byte
\end{itemize}

This format is not easily read so we cannot determine the secret RSA key just from
looking at it.\\
Thus, I transformed this file in a ti-TXT format \cite{titxt}. Each section starts
with a $@$ symbol followed by the start address in hexa. Then, on the next line,
16 bytes of data follow. The first line of \texttt{rsa.txt} translated into ti-TXT
format looks like this:
\begin{lstlisting}
@4000
B2 40 80 5A 20 01 3F 40 A8 58 3E 40 00 11 3D 40
\end{lstlisting}

Using this file format I was able to disassemble it using the package {\em python-msp430-tools}
\cite{pymsp430}, and obtained the corresponding assembly code.

\end{solution}\fi

\item What is the key that you found? Which hints helped you to
locate/extract it? Verify that you found the correct key. Describe your
verification procedure!
\ifsolution\begin{solution}
First of all, I looked for the modulus $N$ in the memory dump file in order to
see its exact address and how is it used. I was able to find it residing at address
{\tt 0x406c} (I also had to take into account the possible endianess of the memory).\\
After that, I checked the assembly code where the address {\tt 0x406c} is used and, of course,
I found an operation of the form:
\begin{lstlisting}
move  &0x406c #0x1160
\end{lstlisting}

I deduced that the same kind of operation has to be done for the secret exponent $D$, so I
did a search on the assembly code and obtained the following possible memory address candidates:
{\tt 0x405c}, {\tt 0x404c}, {\tt 0x403a}, {\tt 0x40e4} and {\tt 0x403a}.\\
All I had to do was look through these addresses and check whether the 14 or 16 byte value starting
from that address is the secret key. For this purpose I implemented three functions in Python:
a logarithmic multiplication for encrypting and decrypting, one for collecting the key from a certain
address and one that looped through the base addresses and checked the key. I had chosen a plaintext
then I encrypted it with the public key, tried to decrypt it with the candidate private keys and verified
if the decrypted ciphertext is equal to the plaintext.\\

\begin{lstlisting}[language=Python]
def mul(a, d, n):
     rez = 1
     while d > 0:
         if d % 2 == 1:
             rez *= a
             rez %= n
         a *= a
         a %= n
         d /= 2
     return rez


def getValue(offset):
  s = ""
  for i in xrange(14, -1, -2):
    s += l[offset - start + i + 1] + l[offset - start + i]
  return s

def checkAddresses(base):
  for b in base:
    print mul(c, int(getValue(b), base=16), n)
\end{lstlisting}

The private key that I found is {\tt 0x114 6bd0 7f0b 74c0 86df 00b3 7c60 2a0b}, located
at address {\tt 0x405C}.
I also checked it with other encrypted plaintexts and verified that it correctly
decrypted them.

\end{solution}\fi

\end{enumerate}

\begin{thebibliography}{1}
\bibitem{eulerth} \href{http://en.wikipedia.org/wiki/Euler's_theorem}{Euler's Theorem}.
\bibitem{ihex} \href{http://en.wikipedia.org/wiki/Intel_HEX}{IHEX File Format}.
\bibitem{titxt} \href{http://manpages.ubuntu.com/manpages/precise/man5/srec_ti_txt.5.html}{ti-TXT File Format}.
\bibitem{pymsp430} \href{https://pythonhosted.org/python-msp430-tools/asm/commandline.html}{Python MSP430 Tools}
\end{thebibliography}

\end{document}
