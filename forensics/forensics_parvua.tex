% IMPORTANT: add or remove (comment out) the boolean '\solutiontrue' below to
% create the solution document or the exercise document respectively.
% First we create the switch to make either the exercises or the solutions
\newif\ifsolution\solutionfalse
% To create the solution uncomment '\solutiontrue'
\solutiontrue

\documentclass[a4paper,11pt]{article}

\title{System Security,
\ifsolution Solution \else \fi
Forensics}


\include{author}


\usepackage[T1]{fontenc}
\usepackage{ae, aecompl}
\usepackage{a4wide}
\usepackage{boxedminipage}
\usepackage{url}
\usepackage{graphicx}
\usepackage{enumerate}
\usepackage{listings}
\usepackage{comment}
\usepackage{ucs}
\usepackage[utf8x]{inputenc}
\usepackage[english]{babel}

\author{Andrei Pârvu}

% Some useful commands and environments
\usepackage{framed}
\newenvironment{solution}%
{\par{\noindent\small\textit{Solution:}}\vspace{-12pt}\begin{framed}}%
{\end{framed}\par}

\begin{document}
\maketitle

\noindent Please download the \textbf{forensics} archive from the course web-page to your home
folder in the VM and unpack it with password 0000.

\section{Recovering Deleted Data}
The goal of this exercise is to give you hands-on experience on recovering
data from a file system. We created a loop-back device (fsimage in the folder
forensics), wrote two images (both in the JPEG format) to it and finally, deleted
the images.  \newline

\noindent In this exercise, you will have to try and extract (by writing a
program/script) the deleted images from the file system using some background
knowledge on file systems and format (syntax and structure) of JPEG
images (\url{http://en.wikipedia.org/wiki/JPEG}). \newline


\noindent Your solution must include:

\begin{itemize}
\item What information about JPEGs did you need to recover the images?
\ifsolution\begin{solution}
The information needed for extracting the images was related to the format of the
JPEG file, specifically the bytes that signal the start of the image (\texttt{0xFF 0xD8})
and the bytes that signal the end of the image (\texttt{0XFF 0XD9}).
\end{solution}\fi

\item The images that you recovered
\ifsolution\begin{solution}
The two images that I recovered are shown in Figure~\ref{fig:password} and Figure~\ref{fig:topsecret}.

\end{solution}\fi

\item Clean, commented code (as text in your solution) that you used to
  recover the image
\ifsolution
\begin{solution}
\begin{lstlisting}[language=C]
#include <unistd.h>
#include <stdio.h>
#include <sys/types.h>
#include <sys/stat.h>
#include <fcntl.h>

#define N 1048576 // ls -l fsimage showed the size of the file

unsigned char bytes[N];
unsigned char v[2][N];
int len[2];

void do_write(unsigned char name[], int poz) {
  // write result to file
  int f1 = open(name, O_RDWR);

  write(f1, v[poz], len[poz]);
  close(f1);
}

int main() {
  // open the image and read it
  int fd = open("forensics/fsimage", O_RDONLY);
  int i, j;
  int cnt = 0;
  int t = read(fd, bytes, N);
  printf("%d\n", t);

  int s = 0;
  for (i = 0; i < N; i++) {
    // look for start of the jpeg file, 0xff 0xd8
    if (bytes[i] == 255 && bytes[i + 1] == 216) {
      for (j = i; j < N; j++) {
        // save the bytes that form the image
        v[cnt][j - i] = bytes[j];
        // look for the end of the jpeg file, 0xff 0xd9
        if (bytes[j] == 255 && bytes[j + 1] == 217) {
          v[cnt][j + 1 - i] = 217;

          // save the length and start looking at the next
          // position in the image
          len[cnt] = j + 1 - i + 1;
          i = j + 2;
          cnt++;
          break;
        }
      }
    }
  }

  do_write("topsecret.jpg", 0);
  do_write("password.jpg", 1);

  return 0;
}
\end{lstlisting}
Note: the program was specifically designed to recover $2$ images. If the number of images was
unknown then it would have been necessary to dynamically allocate the vectors with the image contents.
\end{solution}
\fi


\item Finally, a rationale (short description) behind why you were able to
  recover the data.
\ifsolution
\begin{solution}
The start of file bytes (\texttt{0xFF 0XD8}) and end of file bytes (\texttt{0xFF 0xD9}) delimited
the content of the image so it could be read easily. This could have been prevented if the the
addresses were randomized and the images were not located in a continuous memory layout.
\end{solution}
\fi

\end{itemize}



\section{Practical forensics exercise}

\begin{enumerate}[(a)]

\item Background story\\
  In this practical exercise, we consider the following use case: A
  company called Green Inc. just hired a new employee, Ms. Eve Evil. One
  day after starting in her new job, Ms. Evil quits and
  disappears. The head of the security department, Mr.  Gullible,
  slightly suspects that Ms. Evil tried to steal data from the
  company. He makes a 1:1 copy of the Linux home directory of Ms. Evil
  on her workstation and sends this image to you to find out what
  Ms. Evil did on her only day of work.

\item Material\\
  In the directory \texttt{forensics} inside the VM you will find the image of the
  home file system of Ms. Evil. Set up the image as a
  loopback device as root user: {\tt sudo mount -o loop eve\_home mpoint}.  You
  will now find Ms. Evil's original home directory in the mpoint directory.

\item Documentation\\
  To prove that Ms. Evil indeed did something illegal, as much
  information as possible should be gathered from the image. Be
  creative in finding clues and try to reconstruct Ms. Evil's actions
  to convict her.

\item What information does Eve Evil illegally send out of the company? Where did you find this?
\ifsolution\begin{solution}
First of all, to fully inspect the image of Eve's home folder, I ran the command \texttt{ls -a -R .}.
I first looked in the 'thunderbird' folder, where in the INBOX folder from the GMAIL account
I saw the welcome email from Gary Green which contained confidential information: 'Green\_position.pdf'
and 'Green\_business\_report.pdf'. These are probably the documents Eve stole.
\end{solution} 
\fi

\item Which communication channels does she use? What are the credentials for these channels?
\ifsolution\begin{solution}
Continuing to explore the folders, I encountered the 'jabber' folder (which is an instant messaging
program). I checked out the accounts and logs for it and found that Eve (gmail account eveevil82@gmail.com
with password forensic82) had talked to Mallory (noleti@gmail.com) about the confidential documents.
She will send one via an ftp server and another one via mail and also that we will run a certain script
from Mallory.\\
Continuing with the 'gftp' folder I found the logs from the ftp communication and I determined that Eve:
\begin{itemize}
  \item has connected to 'evil.com:21'
  \item used the username 'eve'
  \item used the password 'xxxx'
  \item stored a file 'br.png'.
  \item retrieved the file 'rootkit.sh'
\end{itemize}

\end{solution} 
\fi

\item What is transmitted over each of the channels?
\ifsolution\begin{solution}
Logging in Eve's gmail account in the Sent Items folder I found an email sent to jemand@gmx.ch containing
the 'Green\_position.pdf' confidential file.\\
Regarding the 'br.png' file uploaded on the ftp server, I used 'ack' (which is a program used to scan all the
files in a folder for a regex) and found in the 'bash\_history' file that Eve changed the name of 'Green\_business\_report.pdf' to
br.png before sending it (also 'Green\_position.pdf' was renamed 'bp.bin').
Thus, I was able to determine how the two confidential documents were sent and to whom.
\end{solution} 
\fi

\item She runs a special program. What is it and what does it do?
\ifsolution\begin{solution}
The program used is called 'rootkit.sh', to which Eve added execution permissions and executed it (this can
also be found in 'bash\_history').
\end{solution} 
\fi

\end{enumerate}


\begin{figure} \center
  \includegraphics[width=0.4\linewidth]{password.jpg}
  \caption{password.jpg}
  \label{fig:password}
\end{figure}
\begin{figure} \center
  \includegraphics[width=0.4\linewidth]{topsecret.jpg}
  \caption{topsecret.jpg}
  \label{fig:topsecret}
\end{figure}

\section{Acknowledgements}
This exercise was designed with the help of Claudio Marforio
(maclaudi@inf.ethz.ch).

\end{document}
