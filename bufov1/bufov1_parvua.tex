% IMPORTANT: add or remove (comment out) the boolean '\solutiontrue' below to
% create the solution document or the exercise document respectively.
% First we create the switch to make either the exercises or the solutions
\newif\ifsolution\solutionfalse
% To create the solution uncomment '\solutiontrue'
\solutiontrue

\documentclass[a4paper,11pt]{article}
\title{System Security,
\ifsolution Solution \else \fi
Buffer Overflow 1}

../author.tex

\usepackage[T1]{fontenc}
\usepackage{ae, aecompl}
\usepackage{a4wide}
\usepackage{boxedminipage}
\usepackage{url}
\usepackage{graphicx}
\usepackage{enumerate}
\usepackage{hyperref}
\usepackage{textcomp}
\usepackage{listings}
\usepackage{ucs}
\usepackage[utf8x]{inputenc}
\usepackage[english]{babel}

\author{Andrei Pârvu}

% Some useful commands and environments
\usepackage{framed}
\newenvironment{solution}%
{\par{\noindent\small\textit{Solution:}}\vspace{-12pt}\begin{framed}}%
{\end{framed}\par}


\begin{document}
\maketitle

\section*{Introduction}

In this assignment you will build exploits for the binary program {\tt vulnapp}.
There are two buffer overflow attacks to perform on the same binary program. You
have already used {\tt vulnapp} in the previous assignment. You find the files
in the folder \texttt{bufov1}. To setup the folder, run \texttt{make} and
enter the syssec password when prompted.

The type of buffer overflow attacks that we propose to do are generally known as
return-to-libc attacks. In such an attack, the main idea is to overwrite the
return address with the address of some function in the libc library (any C
program dynamically links to this library upon execution), pass the correct
parameters to that function and make it execute. The advantage of this attack is
that it can be successfully executed on operating systems with non-executable
stack.

\subsection*{Protection mechanisms}

As you may expect, operating systems already provide a number of measures to
prevent such an attack from being easily successful. One such measure is library
address randomization, that is a part of \textit{Address Space Layout
Randomization} (ASLR). The method randomizes the addresses to which dynamically
linked libraries such as libc are loaded. This brings an additional protection
given that the attacker must predict the function address. In order to make this
exercise easier for you, we suggest to disable the virtual address
randomization. This will ensure a stable virtual address space across multiple
executions. Turning on/off the randomization in a \textit{root shell} is as
follows (off=0, on=1/2):

\begin{verbatim}
echo 0 > /proc/sys/kernel/randomize_va_space
echo 1 > /proc/sys/kernel/randomize_va_space
\end{verbatim}

Inside the VM we have already turned off this randomization.

The compiler can protect the binary by integrating stack canaries. We have
disabled the use of these canaries through the compiler flag
\texttt{-fno-stack-protector} to make this exercise easier. Therefore the binary
has no stack canaries.

However, the stack of the binary is marked non-executable, meaning we cannot
execute code from the stack.

\subsection*{Building \& executing exploits}

Your exploit user input will typically contain byte values that do not
correspond to the ASCII values for printing characters. We recommend using {\it
perl} or {\it python} to supply your exploits on the command line argument. Here
is an example on how call vulnapp with the ASCII characters ``AAA'' followed by
0xf05effbf as argument:\\ \textsf{ bash> ./vulnapp ''\textasciigrave perl -e
'printf ''A'' x 3 . ''\textbackslash xf0\textbackslash x5e\textbackslash
xff\textbackslash xbf''$\:$'\textasciigrave''} or \\ \textsf{ bash> ./vulnapp
''\textasciigrave python2.7 -c 'from struct import pack; print(''A''*3 +
pack(''I'', 0xf05effbf)$\:$'\textasciigrave ''}

This way, you can fill the buffer with arbitrary characters, except zero bytes,
and then add your actual exploit input that contain specific addresses in
hexadecimal format.

\section*{Attack 1: Execute your favorite shell}
\emph{We have collected hints for both exercises at the end of this
  document!}

Your task is to get {\sc vulnapp} to execute a favorite shell of yours (e.g.,
/bin/bash, /bin/sh, /bin/zsh).

To perform the attack you have to supply an exploit string that overwrites the
stored return pointer in the stack frame for {\tt cpybuf} with the address of
the system() function from the libc library and provide the necessary argument.
This function allows you to execute any program (e.g., system("/bin/sh")). To
provide arguments to the system function, you have to prepare the stack to look
like before a regular function call. Think about how the stack layout has to be.
\textit{Hint:} It is usually helpful to sketch it. As the system call expects a
pointer to a string as argument, you will have to find a string, which matches
the path of the file you want to execute. Luckily for you, the vulnapp
application allows you to place a string of your choice in the memory as
follows.

\begin{verbatim}
bash> ./vulnapp Hello
bash>  Type some text:
bash>  /bin/sh
\end{verbatim}

Note that your exploit string may also corrupt other parts of the
stack state. In order to make the program terminate safely you will
need to put on the stack the address of the function exit() to
properly terminate the execution. Here is a list of steps you may
consider following to perform the attack:

\begin{enumerate}
\item Recall the vulnerable buffer length from the previous exercise.
  \item Find the addresses of system() and exit().
  \item Find the address that points to your typed text containing the shell to
  execute.
  \item Understand how to position these addresses in your exploit.
  \item Build and test the exploit.
  \item Check if you escalated your privileges, e.g. using \texttt{id}.
\end{enumerate}

\textbf{Important:} While you can use gdb to prepare these attacks, you should
perform the actual attack against the normally running executable which you
started as normal user(syssec).

\ifsolution
\begin{solution}
In order to execute this attack, I had to determine 3 addresses:
\begin{itemize}
  \item address of the \texttt{system} function. By running \texttt{print system} in gdb, I obtained
  \texttt{0xb7e3afe0}.
  \item address of the \texttt{exit} function: \texttt{0xb7e2eb50}
  \item address of the string given to the program at stdin: \texttt{0x80499c0}.
\end{itemize}

The input provided to the program consisted of 28 bytes (24 the buffer + 4 the saved $ebp$) equal to $A$
(zero would have been a bad idea, because it's the string terminator), followed by the \texttt{system}
address (as the saved $eip$), followed by the \texttt{exit} address (as the saved $eip$ when executing
\texttt{system}), the address of the input (in my case, \texttt{/bin/sh}) and a final 0 (argument for exit, it's presence
is assured by \texttt{strcpy}). So, the stack looks like this, after the \texttt{strcpy} call:
\begin{lstlisting}
        [ 0x080499c0 ] <- argument for system
        [ 0xb7e2eb50 ] <- saved eip / exit
        [ 0xb7e3afe0 ] <- saved eip / system
        [ 0x41414141 ] <- saved ebp 'AAAA'
        [ 0x41414141 ] <- 'AAAA'
        [ 0x41414141 ] <- 'AAAA'
        [ 0x41414141 ] <- 'AAAA'
        [ 0x41414141 ] <- 'AAAA'
        [ 0x41414141 ] <- 'AAAA'
        [ 0x41414141 ] <- 'AAAA'
        [ 0xbffffbb4 ] <- argument to 'strcpy',
ESP  -> [ 0xbffff96c ] <- address of location on stack
\end{lstlisting}

The execution can be seen in Figure~\ref{fig:system}; \texttt{geninput} is a program that generates
a $28$ byte padding ($24$ the size of the buffer + 4 bytes saved $ebp$). The \texttt{vulnapp} exe is from
the previous assignment.\\
The main use of gdb was to explore the stack during debugging (with the command \texttt{x/10x \$sp})
and to check the saved $eip$ using \texttt{info frame}.

\end{solution}\fi

\begin{figure}[H] \center
  \includegraphics[width=1\linewidth]{pics/system.png}
  \caption{\texttt{system} attack}
  \label{fig:system}
\end{figure}

\section*{Attack 2: Execute the shell from environmental variables}

Not every program might allow you to place a string in memory through an extra
input as vulnapp. So, this attack is similar to the previous one with the
difference that you will need to find the path to the shell i.e., "/bin/sh" in
the environmental variables.  These variables are automatically loaded in the
program stack upon execution. Therefore you are not dependent on the additional
input.

For this attack, you do not need to supply any information when the program asks you
to type some text as it will not be needed. In difference to the previous attack,
you just need to find the path to the shell program already included in the program
space after loading the vulnapp program.

The environmental variables are provided to the program as an argument. You can
either use the existing \texttt{SHELL} variable, you can modify with
\texttt{export SHELL=/bin/sh}, or create a new variable as \texttt{export
NEWVAR=/bin/sh}. To find the correct address of the variable, you have multiple
options\dots

\ifsolution\begin{solution}
To find the address of the \texttt{SHELL} environment variable I used the \texttt{getenv}
function which returns a pointer to the variable, in my case \texttt{0xbffffc64}.
By adding it instead of the address of the stdin parameter of the program, it launched
the corresponding shell (Figure~\ref{fig:var}).\\
In gdb, the main way to debug and find the address of the environment variabile was to use
the command \texttt{print *((char**)environ)} and loop through all the variabiles by
incrementing \texttt{environ}.

\end{solution}\fi

\begin{figure}[H] \center
  \includegraphics[width=1\linewidth]{pics/var.png}
  \caption{Environment variable attack}
  \label{fig:var}
\end{figure}

\section*{Further options}
Name another option where the argument for the call to system could be stored.
\ifsolution\begin{solution}
Another option would be to pass it as a second parameter to our program, if the program
doesn't check the number of arguments that it receives.
\end{solution}\fi

\noindent
Name another option to terminate the program without a segmentation fault and
without jumping directly to exit().  
\ifsolution\begin{solution}
Another possibility would have been using the \texttt{exec} family functions. This
offers an advantage, because it replaces the current process with the one given as argument
(opposed to \texttt{system}, which launches a new process and waits for its completion).\\
For my test, I used the \texttt{execl} function, which has an address of \texttt{0xb7eb3250}.
In this case, you don't need to supply a good address as the saved $eip$ for \texttt{execl},
it can be rubbish (\texttt{AAAA}, in my case). The results can be seen in Figure~\ref{fig:execv}.

\end{solution}\fi

\begin{figure}[H] \center
  \includegraphics[width=1\linewidth]{pics/execv.png}
  \caption{\texttt{execv} attack}
  \label{fig:execv}
\end{figure}

\section*{Expected Deliverables and Some Advice/Help}

First, we point out that this exercise is individual and we expect individual
submissions. Your submission should contain the important gdb commands used and
their output for each step towards successfully performing the attack. It should
also include a brief explanation after some major steps to summarize your
findings.

\end{document}
