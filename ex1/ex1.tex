% IMPORTANT: add or remove (comment out) the boolean '\solutiontrue' below to
% create the solution document or the exercise document respectively.
% First we create the switch to make either the exercises or the solutions
\newif\ifsolution\solutionfalse
% To create the solution uncomment '\solutiontrue'
\solutiontrue

\documentclass[a4paper,11pt]{article}
\title{System Security, Autumn 2014}

\ifsolution\author{Andrei Parvu}
\else\author{Exercise 1}\fi

../author.tex

\usepackage[T1]{fontenc}
\usepackage{ae, aecompl}
\usepackage{a4wide}
\usepackage{boxedminipage}
\usepackage{url}
\usepackage{graphicx}
\usepackage{enumerate}

% Some useful commands and environments
\newcommand{\includeonlyinsolution}[1]{\ifsolution#1\fi}
\usepackage{framed}
\newenvironment{solution}%
{\par{\noindent\small\textit{Solution:}}\vspace{-12pt}\begin{framed}}%
{\end{framed}\par}

\begin{document}
\maketitle

\includeonlyinsolution{
\subsection*{General remarks}
Here are a few remarks that are always a good to remember when you are doing the
exercises. These remarks apply to all the exercises but will just be included
here as a friendly advise.
\begin{itemize}
\item Be specific. What is important? It is better to explain the ``cause'' of a
problem in detail, than to only mention a ``consequence''.
\item Answer all questions as thoroughly as you can.
\item Google is your friend.
\item Many spelling/formulation errors give a careless impression.
\end{itemize}
}

\section{Cryptography Basics}
This section will review some basic security concepts.
\begin{enumerate}[(a)]
\item Explain the terms \emph{Integrity}, \emph{Confidentiality},
  \emph{Authentication}, \emph{Authorization}, and 
  \emph{Denial of Service}.
\includeonlyinsolution{\begin{solution}
\begin{itemize}
  \item \emph{Integrity}: Prevention of unauthorized modification of information \cite{CompSec}
  \item \emph{Confidentiality}: Prevention of unauthorized disclosure of information \cite{CompSec}
  \item \emph{Authentication}: Verifying the identity of a given entity.
  \item \emph{Authorization}: Specifying acces rights of a given entity. Required prior authentication.
  \item \emph{Denial of Serice}: Prevention of authorized access to resources
  or the delaying of time-critical operations \cite{ISO7498}.
\end{itemize}
\end{solution}}

\item Why do we need both symmetric and
   asymmetric cryptography?
  \includeonlyinsolution{\begin{solution}
  Symmetric encryption has some advantages over asymmetric encryption:
  \begin{itemize}
    \item its keys are simpler and faster to generate (usually random) and do not need to have any special properties.
    \item symmetric encryption keys are smaller than asymmetric ones. This is caused by the fact that
    asymmetric keys depends on the problem of integer factorization, which, althugh difficult, is faster
    than the brut force method used for symmetric keys.
  \end{itemize}
  The main disadvantage of symmetric encryption is its scalability: in a group of $n$ users, in which
  each pair of users desires to communicate securely, a total of $O(n^2)$ keys are necessary. In a
  similar group in which asymmetric keys are used, a total of $O(n)$ keys are used.\\
  Because of these advantages and disadvantages, a combination of both symmetric and asymmetric
  keys is used for communication:
  \begin{itemize}
    \item the authentication is done using public-private keys.
    \item the ensuing communication is done using symmetric keys.
  \end{itemize}
\end{solution}}

\item Describe the following five attacks \emph{Ciphertext only} / \emph{Known plaintext} /
  \emph{Chosen plaintext} / \emph{Chosen ciphertext} / \emph{Chosen
    cipher- and plaintext}. Rank these attacks in terms of their potential to
  be successful.

\includeonlyinsolution{\begin{solution}
\begin{itemize}:
  \item \emph{Ciphertext only}: the attacker has access to a set of ciphertexts, and tries to obtain
  information about the corresponding plaintexts or about the key. Ciphertext only attack, especially
  frequency analysis, was used to break early ciphers, such as transposition cipher or Vigenere cipher.
  \item \emph{Known plaintext}: the attacker has samples of both the ciphertext and the corresponding
  ciphertext and is trying to obtain the key.
  \item \emph{Chosen plaintext}: the attacker has the capability to encrypt plaintexts and obtain
  the corresponding ciphertexts, its goal being to obtain information about the key.
  \item \emph{Chosen ciphertext}: the attacker has the capability of choosing a ciphertext and obtaining
  its decryption. It is presumed that this ability is no longer available when the attacker obtains
  the ciphertext it wants to decrypt.
  \item \emph{Chosen cipher and plaintext}: A combination between the previous two.
\end{solution}}
\end{enumerate}

\section{Locking System}
A car manufacturer wants to equip his cars with a new locking system which works in
the following way: The system is composed of a small (tamper-proof) sender embedded
in the car key and a receiver in the car. When the sender goes close to the
receiver, it sends a radio frequency signal. The signal consists of 128-bit string which is the
actual cryptographic key and which will be checked by the receiver. The range of the
sender is about $20m$.

The cryptographic key length renders a brute force attack impossible and the sender
can be considered tamper-proof.
\begin{enumerate}[(a)]
\item The system seems to be as secure as a classical lock. Do you see another
possibility to break the security of the system?
\includeonlyinsolution{\begin{solution}
WRITE HERE.
\end{solution}}

\item What can be done to avoid the attack you just described?
\includeonlyinsolution{\begin{solution}
WRITE HERE.
\end{solution}}

\item Would your solution solve the ``relay attack'' problem as described in
  Francillion A. et. al. in their work ``Relay
  Attacks on Passive Keyless Entry and Start Systems in Modern Cars'' (\url{http://eprint.iacr.org/2010/332.pdf})?
\includeonlyinsolution{\begin{solution}
WRITE HERE.
\end{solution}}
\end{enumerate}

\section{Side Channel Attacks Basics}
This section will review some important concepts underlying the side channel
attacks described in the lecture. 
\begin{enumerate}[(a)]
\item What is a side channel attack?
\includeonlyinsolution{\begin{solution}
WRITE HERE.
\end{solution}}
\item What is the difference between simple and differential side channel
  (e.g., power, timing) analysis?
\includeonlyinsolution{\begin{solution}
WRITE HERE.
\end{solution}}

\item Computers and screens emit electromagnetic waves which can be measured and
analyzed from a distance. This might allow an attacker to draw conclusions from the
processed data. On which parameters does the maximum ``attack--distance'' from the
receiver to the emission source depend?
\includeonlyinsolution{\begin{solution}
WRITE HERE.
\end{solution}}


\end{enumerate}

\section{Simple Power Analysis Attacks on RSA}
\begin{enumerate}[(a)]

\item The ``Square and Multiply'' algorithm is commonly used to implement
  modular exponentiation. The algorithm computes the exponentiation by a
  series of squarings and multiplications. It processes the exponent bitwise,
  and for each bit, a squaring is executed. If the current bit of the exponent
  is $e_i=1$, the intermediate result is multiplied with the base $a$. \\
  What makes RSA implementations that use square and multiply (in its native
  form that we just described) vulnerable to a side channel
  attack?
\includeonlyinsolution{\begin{solution}
WRITE HERE.
\end{solution}}


\item Which platforms are most vulnerable to power-based side-channel attacks and why?
\includeonlyinsolution{\begin{solution}
WRITE HERE.
\end{solution}}
\end{enumerate}


\section{Cache-timing Attack on AES}
\begin{enumerate}[(a)]
\item What is the vulnerability exploited in the attack described at the
  lecture? Does the vulnerability lie in the design or in the implementation of AES?
\includeonlyinsolution{\begin{solution}
WRITE HERE.
\end{solution}}

\item For a $k$ byte key, how many messages would you require in theory for
  this attack? You may assume a noiseless/ideal measurement system.
\includeonlyinsolution{\begin{solution}
WRITE HERE.
\end{solution}}
\end{enumerate}

\begin{thebibliography}{1}

  \bibitem{ISO7498} Interntional Organization for Standardization
    {\em Basic Reference Model for Open Systems Interconnection:
    Security Architecture}  1989.
  \bibitem{CompSec} Dieter Gollmann {\em Computer Security}, 2011
\end{thebibliography}
\end{document}

%%% Local Variables: 
%%% mode: latex
%%% TeX-master: t
%%% End: 
